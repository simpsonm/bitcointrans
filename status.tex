\documentclass[12pt]{article}

\setlength{\oddsidemargin}{-0.125in}
\setlength{\topmargin}{-0.5in} \setlength{\textwidth}{6.5in}
\setlength{\textheight}{9in}
\usepackage{rotating}
\setlength{\textheight}{9in} \setlength{\textwidth}{6.5in}
\setlength{\topmargin}{-40pt} \setlength{\oddsidemargin}{0pt}
\setlength{\evensidemargin}{0pt}
\usepackage{adjustbox,lipsum}
\setlength{\textheight}{8.5in} \setlength{\textwidth}{6.5in}
\setlength{\topmargin}{-36pt} \setlength{\oddsidemargin}{0pt}
\setlength{\evensidemargin}{0pt} \tolerance=500
\renewcommand{\baselinestretch}{1.5}

\usepackage{amsmath, amsfonts}
\usepackage{bm, bbm}
\usepackage{graphics, graphicx, color, epsf}
\usepackage[authoryear]{natbib}
\usepackage{xr, zref, hyperref}

\begin{document}
\section{Introduction}
This note describes the current status of the project. In particular, is states the first model we used and discusses its limitations. Then it discusses what to do about those limitations, including whether we should use the second model.

\section{The Block Size Model}
The block size model is the original model we developed. I present a slightly different version of it here with a different measurement error model and a more general process model. Let $k=1,2,\dots,K$ denote blocks in order of arrival on the block chain. Then we model two key quantities: the arrival times of the each block in minutes, denoted by $\tau_k$, and the size of the block in MB, denoted by $d_k$. We directly observe $d_k$, but not $\tau_k$. Rather, we observe $t_k$, a noisy measurement of $\tau_k$ that is restricted to be larger than the median of $\bm{t}_{k-1:11} = (t_{k-11}, t_{k-10}, \dots, t_{k-1})$, which we denote by $m_k$. The measurement error is due to the fact that different computers mine different blocks, and they may not not have perfectly synced internal clocks. As a result, the block chain protocol rejects all blocks that have a reported arrival time that is too early.

In the arrival time measurement error model, we assume that the observed arrival times are normally distributed, centered on the true arrival time, but truncated from below at $m_k$. That is
\begin{align}
t_k | \bm{t}_{k-1:11}, \tau_k, \sigma^2 &\stackrel{ind}{\sim}\mathrm{N}(\tau_k, \sigma^2)1(t_k>m_k) && \mbox{(arrival time measurement model)} \label{eq:error}\\
\intertext{for $k=1,2,\dots,K$. Next, define the block inter-arrival times, $\delta_k = \tau_k - \tau_{k-1}$ for $k=1,2,\dots,K$ where $\tau_0\equiv 0$. The bitcoin protocol is designed so that the blocks follow a Poisson process with intensity $1/10$ per minute. This implies that the inter-arrival times are exponentially distribution with with mean $10$ minutes, i.e.}
\delta_k = \tau_k - \tau_{k-1} &\stackrel{iid}{\sim}\mathrm{Exp}(1/10) && \mbox{(inter-arrival time process model)}\label{eq:arrival}
\end{align}
for $k=1,2,\dots,K$.

Next, we need a model for the block sizes, i.e. the $d_k$s. The longer it takes for a block to arrive, the more transactions and therefore the more transaction data it should accumulate. Let $D(\tau)$ denote the total accumulated transaction data in the block chain. Then $D(\tau)$ we assume that $D(\tau)$ is a nondecreasing L\'{e}vy process, i.e. that $\{D(\tau): \tau > 0\}$ is continuous in $\tau$ with independent, stationary increments. There are many options here, but a convenient L\'{e}vy process is the gamma process, defined by
\begin{align*}
D(\tau + \delta) - D(\tau) &\sim \mathrm{G}(\Psi(\tau + \delta) - \Psi(\tau), \theta)
\end{align*}
for any time $\tau>0$ and increment $\delta > 0$ where $\Psi(\tau) = \int_0^\tau\psi(u)du$, and $\psi(u)>0$ is the intensity function of the  process. We use the shape/rate parameterization of the gamma distribution so that $\mathrm{E}[D(\tau + \delta) - D(\tau)] = [\Psi(\tau + \delta) - \Psi(\tau)]/\theta$. To fully specify this process model we need to specify the intensity function, but for now we will leave that aside. Given the gamma process for $D(\tau)$, we can define the block size process model. Let $D_k = D(\tau_k)$, $d_k = D_k - D_{k-1}$, and $\psi_k = \Psi(\tau_k)$ for $k=1,2,\dots,K$ where we set $D_0 = D(0) = 0$. Then we assume
\begin{align}
d_k | \psi_k, \theta &\stackrel{ind}{\sim}\mathrm{G}(\theta\psi_k, \theta) && \mbox{(block size process model)}\label{eq:size}
\end{align}
for $k=1,2,\dots,K$. We parameterize the intensity function and the gamma process so that $\mathrm{E}[d_k|\psi_k,\theta] = \psi_k$, so that the intensity function is $\theta\psi(u)$ and $\psi(u)$ defines the mean of $d_k$'s gamma distribution.

Equations \eqref{eq:error} - \eqref{eq:size} define the block size model. To complete this model we must specify the intensity function, $\psi(u)$, and priors for all of the unknown parameters. In the original version of the paper we only considered a constant intensity $\psi(u) = \gamma > 0$, though we can also define an increasing intensity $\psi(u) = e^{\beta_0 + u\beta_1}$. In principle we can define any intensity function of the form $\psi(u) = e^{\bm{z}(u)'\bm{\beta}}$ where $\bm{z}(u)$ is a vector of basis functions of time, and $\bm{\beta}$ is a vector of regression coefficients. However, we need to be able to compute the integral $\int_{\tau}^{\tau + \delta}\psi(u)du$ in closed form, which restricts the form of $\psi(u)$. One possibility that I have not explored that may be more flexible is $\psi(u) = \sum_{p=1}^P\beta_pu^{\alpha_p - 1}$ with $\alpha_p, \beta_p\geq 0$ for $p=1,2,\dots,P$. In this case, then $\Psi(\tau) = \int_0^\tau\psi(u)du = \beta_p\tau^{\alpha_p}/\beta_p$. If $\alpha_p < 1$ then the $p$th term of $\psi(u)$ decays in $u$, while if $\alpha_p > 1$ it increases in $u$ and if $\alpha_p = 1$ it is constant in $u$. Strictly speaking we require $\alpha_p > 0$ since if $\alpha_p = 0$ then $\Psi(0) = -\infty$, while if $\alpha_p > 0$ then $\Psi(0) = 0$, which is what we require.

\subsection{Problems and solutions}
The big problem with this model, which we identified after we got feedback on the first draft, is that the block size process model does not take into account that excess transactions that arrive during one block spill over into the next block. So all observed block sizes will be lower than the block size limit for that particular block. For example is every block has a 1 MB block size limit, we would never observe a block with more than 1 MB of transactions, even when the limit is routinely being reach. What $d_k$ represents is not the total amount of transaction data that was {\it submitted} to the block chain after block $k-1$ arrived but before block $k$, instead it represents the total amount of transaction data {\it accepted} by the block chain in block $k$.

There is information in the observed data about the spill over. When a block's size is close to the cap, then we know there's a decent chance that at least one transaction was pushed into the next block. There are a couple of problems with this, however: 1) diffwerent bitcoin miners implement different caps, and 2) transaction data comes in discrete chunks. As a consuquence of 2), a block's size can be strictly smaller than the cap and still have spill over transactions, so determining whether a block hit the cap is not as straightforward as seeing whether the block size is the same as the cap size. Further, because of 1) we do not know the cap except for a maximum value, e.g. 1 MB throughout much of bitcoin's history.

We tried to solve this problem with the second model we built. I won't recapitulate it here, but basically we replaced Equation \eqref{eq:size} with 1) a transaction arrival stochastic process, 2) a transaction size model, and 3) incorprated the block size constraints to allow for spillover. This model ended up being very messy and as a result difficult to fit --- I had to write my own MCMC algorithm to fit it. Plus, it seemed to drastically overestimate the probability that a given block has spillover. Because of this, I don't think we should pursue this model.

As an alternative, what we could do is stick with the model we have above and fit a separate model using block sizes (in MB) and block transaction counts in order to estimate the size of a typical transaction. Then we can compare the distribution of transaction sizes to the difference between the oberved block sizes and the cap in order to predict when the cap has been reached. We can do the same thing with the {\it predicted} block sizes, which we can obtain from the posterior predictive distribution of the block size model. This is substantially easier to do.

The transaction data model might look like this. Suppose $x_i$ denotes the amount of data for transaction $i$, then we assume $x_i \stackrel{iid}{\sim}\mathrm{G}(\alpha, \beta)$. But we do not observed the $x_i$s. Instead we observe the $d_k$, the amount of data in block $k$, and $n_k$, the number of transactions in block $k$. The model on the $x_i$s implies then that $d_k \stackrel{ind}{\sim} \mathrm{G}(\alpha n_k, \beta)$. So we can fit this model and get parameter estimates for $\alpha$ and $\beta$, and in particular compute $P(x_i > x)$ --- the probability that a given transaction is size $x$ or large for any $x$. When $x$ is the difference between the block size cap and an observed block size, then this is an estimate of the probability that there was block overflow.

\section{What to do next}
In short, this is what I want to do in order to get this paper into a form that we can submit:
\begin{enumerate}
\item Get updated data from the block chain. (Can we easily do this?)
\item Fit transaction size models. The model I mentioned in the last paragraph of the previous section assumes that transactions {\it sizes} have been constant, on average, over time, and any differences in block sizes is due to the number of transactions. We can investigate this pretty easily.
\item Fit several versions of the block size model --- basically with different choices of $\psi(u)$.
\item Compare predictions of the block size model to reality --- do some one step ahead predictions to see how well we predict block sizes. We can do this in different regions of the blockchain --- e.g. with the old data we have, and also with up-to-the-minute data.
\item Use 1) and 2) in order to estimate the probability of transaction spillover in previous blocks, and new blocks.
\item Write everything up.
\end{enumerate}

Getting these models implemented and fit shouldn't be much work for me. Everything will be in Stan, so it'll mostly be a matter of pressing buttons. The most time consuming part will be writing everything up, including getting references to the relevant literature.
\end{document}